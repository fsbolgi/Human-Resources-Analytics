\section*{Conclusioni}
	In conclusione, in questo progetto abbiamo compreso come è formato il database di Human Resources e quali siano le caratteristiche più importanti per classificare un impiegato.$\newline$
	Fin dall'analisi delle correlazioni tra le variabili, passando per gli algoritmi di clustering k-means e DBscan ed ottenendo ulteriori conferme dalle regole di associazione e dai decision trees, siamo riusciti ad estrarre un marcato gruppo di attributi che descrivono una categoria di lavoratori che lasciano il lavoro. In particolare questi si distinguono per un basso livello di soddisfazione, minore di 0.5, un numero minimo di progetti, ossia 2 o massimo 3, una valutazione inferiore alla media, con valori che oscillano tra 0.4 e 0.6, ed hanno passato tra i 2 e i 4 anni nell'azienda lavorando in media tra le 6 e le 7.5 ore al giorno.$\newline$
	Se quindi l'azienda riscontra un impiegato che presenta queste caratteristiche e desidera che mantenga il proprio posto di lavoro dovrà prendere provvedimenti per cercare, ad esempio, di capire come mai il livello di soddisfazione sia basso oppure perché dopo pochi anni nell'azienda una persona decida di cambiare lavoro.$\newline$
	I decision trees, inoltre, hanno evidenziato la presenza di un’altra categoria di impiegati che lasciano la compagnia. Si tratta di un gruppo di persone che hanno una valutazione superiore alla media, ovvero maggiore di 0.8, che hanno passato  nella compagnia un numero di anni maggiore della media, tra i 4 e i 7, lavorando più della media, 10 o più ore al giorno, e realizzando un numero di progetti molto vicino alla media o superiore ad essa, più di 3.5. Per questa categoria, inoltre, anche il livello di soddisfazione, maggiore di 0.7, è superiore alla media.$\newline$
	Lo scopo dell'analisi era di individuare perchè gli impiegati migliori decidessero di lasciare l'azienda. Nel corso della nostra analisi abbiamo individuato un gruppo di persone che lasciano aventi valutazioni mediocri, e quindi che magari l'azienda non è interessata a tenere, ed uno aventi valutazioni molto alte, che potrebbe coincidere con il target iniziale.