\section*{Introduzione}
\label{sec:intro}
\addcontentsline{toc}{section}{\nameref{sec:intro}}
	Questo progetto consiste nell’analisi del database Human Resources (HR), scaricabile dalla piattaforma Kaggle e contenente informazione sugli impiegati di una compagnia, allo scopo di predire quali impiegati lasceranno l’azienda prematuramente e di comprendere quali sono i principali motivi del loro licenziamento. $\newline$
	La realizzazione di un modello accurato che permetta di prevedere se un impiegato lascerà o no il proprio lavoro costituisce un vantaggio per l’azienda, poiché le permette di individuare le caratteristiche principali degli impiegati che lasciano il posto di lavoro e, soprattutto in caso di dipendenti qualificati e con esperienza, di servirsene per evitarne la perdita, attuando dei provvedimenti interni indirizzati proprio verso la categoria di impiegati interessati. Ad esempio, se dal modello si evidenzia che a lasciare il posto di lavoro prematuramente sono gli impiegati con un salario più basso, l’azienda potrà, se lo ritiene opportuno, intervenire in maniera preventiva proponendo degli aumenti di stipendio agli impiegati interessati.$\newline$
	L’analisi dei dati è suddivisa in quattro fasi principali:\vspace{-0.1cm}
	\begin{enumerate}
		\item Data Understanding: è formato da due fasi, la comprensione dei dati tramite statistiche, valutazione della qualità e misura della correlazione e la preparazione dei dati per le analisi successive (gestione di eventuali missing values e outliers, normalizzazione, eliminazione di variabili ridondanti);\vspace{-0.1cm}
		\item Clustering: consiste nell'applicazione degli algoritmi di clustering K-means, DBSCAN e Agglomerative Hierarchical, al fine di trovare raggruppamenti ottimali dei dati;\vspace{-0.1cm}
		\item Association Rules Mining: ha il compito di individuare i pattern più frequenti e valutare quali siano i più interessanti;\vspace{-0.1cm}
		\item Classification: consiste nello sviluppo di un modello di predizione accurato e affidabile.
	\end{enumerate}
